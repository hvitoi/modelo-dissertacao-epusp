\chapter{Fundamentos}
\label{cha:fundamentos}

\lipsum

\section{Section 1}
\label{sec:industria4}

\lipsum

\begin{figure}[htb]
	\centering
	\includegraphics[width=1\textwidth]{sample.png}
	\caption{Sample image.}
	\label{fig:sample2}
	\fonte{\citeonline{wahlster2013industrie} (adaptado). }
\end{figure}

Conforme listados na \autoref{tab:tabela1}.

\begin{table}[htb]
	\centering
	\footnotesize
	\begin{tabular}{p{3cm}p{12cm}}
		\hline
		\textbf{Princípio}           & \textbf{Descrição}                                                                                                                                                                                                                                                                                                           \\

		\hline
		Interoperabilidade           &
		Capacidade das coisas (máquinas, dispositivos, sensores, pessoas, etc) de comunicarem entre si dentro de um sistema por meio de padrões definidos.                                                                                                                                                                                                          \\

		\hline
		Transparência de informação  &
		Tornar acessíveis informações úteis para os demais dispositivos conectados à rede. Informações do mundo virtual como documentos eletrônicos, desenhos, modelos de simulação; e informações sobre o mundo real, como posição, dados de sensores de temperatura, vibração, etc.                                                                               \\

		\hline
		Descentralização de decisões &
		Permitir a tomada de decisões baseada nas informações coletados pelo próprio dispositivo e dar ao dispositivo autonomia para decidir qual será sua próxima função/operação. Desta forma, um planejamento ou controle central de processos produtivos não se faz essencial e o sistema de produção se torna menos hierarquizado.                             \\

		\hline
		Assistência técnica          &
		Devido à complexidade da produção, com redes complexas e tomada decisões descentralizadas, os seres humanos precisam ser auxiliados por sistemas de assistência de forma a dar compreensibilidade ao processo e às tomadas de decisão necessárias. Os sistemas de assistência devem agregar e tornar visualizáveis as informações de maneira compreensível. \\

		\hline
	\end{tabular}
	\caption{Tabela 1.}
	\label{tab:tabela1}
\end{table}

\subsection{Subsection }
\label{sub:rami4}

\lipsum